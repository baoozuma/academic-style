\documentclass[11pt]{article}

% --- Font và encoding ---
\usepackage[T1]{fontenc}
\usepackage[utf8]{inputenc}

\usepackage{mathpazo}

% --- Cấu hình lề và spacing ---
\usepackage[margin=2.5cm]{geometry}
\usepackage{setspace}
\setstretch{1.2}
\setlength{\parskip}{0.5em}
\setlength{\parindent}{0pt}

% --- Header/footer đẹp ---
\usepackage{fancyhdr}
\pagestyle{fancy}
\fancyhf{}
\fancyhead[L]{\nouppercase{\leftmark}}
\fancyhead[R]{\thepage}

% --- Định dạng đề mục ---
\usepackage{titlesec}
\titleformat{\section}{\Large\bfseries}{\thesection.}{0.5em}{}
\titleformat{\subsection}{\large\bfseries}{\thesubsection.}{0.5em}{}
\titleformat{\subsubsection}{\normalsize\bfseries}{\thesubsubsection.}{0.5em}{}

% --- Toán học và danh sách ---
\usepackage{amsmath, amssymb, amsthm}
\usepackage{enumitem}

% --- Hyperlink ---
\usepackage[colorlinks=true, linkcolor=blue, urlcolor=blue, citecolor=blue]{hyperref}

% --- Gói màu và khung ---
\usepackage[most]{tcolorbox}
\tcbuselibrary{theorems}

% --- Các môi trường định lý có khung + tự động nhãn ---
% --- Các môi trường toán học có khung ---
\newtcbtheorem[number within=section]{theorem}{Theorem}%
  {colback=blue!2!white, colframe=blue!40!black,
   fonttitle=\bfseries,
   boxed title style={enhanced, top=2mm, toprule=1pt},
   title={Theorem},
   breakable,
   before upper={\leavevmode\parindent0pt}
}{th}

\newtcbtheorem[number within=section]{definition}{Definition}%
  {colback=green!2!white, colframe=green!40!black,
   fonttitle=\bfseries,
   boxed title style={enhanced, top=2mm, toprule=1pt},
   title={Definition},
   breakable,
   before upper={\leavevmode\parindent0pt}
}{def}

\newtcbtheorem[number within=section]{lemma}{Lemma}%
  {colback=yellow!20!white, colframe=orange!100!black,
   fonttitle=\bfseries,
   boxed title style={enhanced, top=2mm, toprule=1pt},
   title={Lemma},
   breakable,
   before upper={\leavevmode\parindent0pt}
}{lem}

\newtcbtheorem[number within=section]{example}{Example}%
  {colback=gray!2!white, colframe=gray!50!black,
   fonttitle=\bfseries,
   boxed title style={enhanced, top=2mm, toprule=1pt},
   title={Example},
   breakable,
   before upper={\leavevmode\parindent0pt}
}{ex}


% --- Môi trường chứng minh ---
\renewenvironment{proof}[1][\proofname]{\par\noindent\textit{#1.} }{\hfill$\square$\par}

% --- Thông tin tài liệu ---
\title{\textbf{Linear Algebra}}
\author{
  \textbf{Aleksis Pham} \\ Pure Math, HCMUS \and
  \textbf{Kento Kazuma} \\ Philosophy, Munich \and
  \textbf{Pham Bao} \\ Computer Science, MIT
}

\date{\today}

\begin{document}

\maketitle

\section{Vector Spaces}

\begin{definition}{Vector Space}{1}
A triple $(V, +, \cdot)$ consisting of a set $V$, an addition map
\[
  + : V \times V \to V, \quad (x,y) \mapsto x + y
\]
and a scalar multiplication map
\[
  \cdot : \mathbb{F} \times V \to V, \quad (\lambda, x) \mapsto \lambda x
\]
is called a real vector space if for all $x, y, z \in V$ and $\lambda, \mu \in \mathbb{R}$, the following axioms hold:
\begin{enumerate}[label=\arabic*.]
  \item $(x + y) + z = x + (y + z)$
  \item $x + y = y + x$
  \item $x + 0 = x$
  \item $x + (-x) = 0$
  \item $\lambda(\mu x) = (\lambda \mu)x$
  \item $1x = x$
  \item $\lambda(x + y) = \lambda x + \lambda y$
  \item $(\lambda + \mu)x = \lambda x + \mu x$
\end{enumerate}
\end{definition}

\textbf{Note:} $0$ and $1$ here denote the additive and multiplicative identities in $\mathbb{R}$.

\begin{definition}{Subspace}{1}
A subset $U$ of $V$ is a subspace if:
\begin{itemize}
  \item $0 \in U$
  \item $u, w \in U \Rightarrow u + w \in U$
  \item $a \in \mathbb{F},\ u \in U \Rightarrow au \in U$
\end{itemize}
\end{definition}

\begin{definition}{Sum of Subspaces}{1}
If $V_1, V_2, \dots, V_m$ are subspaces of $V$, their sum is
\[
  V_1 + V_2 + \dots + V_m = \left\{ v_1 + v_2 + \dots + v_m : v_i \in V_i \right\}
\]
\end{definition}

\begin{theorem}
The sum $V_1 + \dots + V_m$ is the smallest subspace of $V$ containing all $V_i$.
\end{theorem}
\begin{proof}
Because $0 \in V_i$, we have $0 + \dots + 0 = 0 \in V_1 + \dots + V_m$. For any $x \in V_i$, we can express it as $0 + \dots + x + \dots + 0$ so $x \in V_1 + \dots + V_m$. Hence $V_i \subset V_1 + \dots + V_m$.

Suppose $U$ is a subspace containing all $V_i$. Let $x \in V_1 + \dots + V_m$, then $x = v_1 + \dots + v_m$ with $v_i \in V_i \subset U$. Since $U$ is closed under addition, $x \in U$. So $V_1 + \dots + V_m \subset U$.
\end{proof}

\begin{definition}{Direct Sum}{1}
The sum $V_1 + \dots + V_m$ is a \textit{direct sum}, denoted $V_1 \oplus \dots \oplus V_m$, if each element of the sum can be written uniquely as $v_1 + \dots + v_m$ with $v_i \in V_i$.
\end{definition}

\begin{theorem}
The sum $V_1 + \dots + V_m$ is a direct sum if and only if the only way to write $0$ as $v_1 + \dots + v_m$ with $v_i \in V_i$ is when all $v_i = 0$.
\end{theorem}
\begin{proof}
Suppose $v_1 + \dots + v_m = 0$ implies $v_i = 0$. Assume $v = x_1 + \dots + x_m = y_1 + \dots + y_m$ with $x_i, y_i \in V_i$. Subtracting gives $(x_1 - y_1) + \dots + (x_m - y_m) = 0$. By hypothesis, each $x_i = y_i$. Thus representation is unique.
\end{proof}

\begin{theorem}
For subspaces $U, W \subset V$,
\[
  U + W \text{ is a direct sum } \Leftrightarrow U \cap W = \{0\}
\]
\end{theorem}
\begin{proof}
(\( \Rightarrow \)) Suppose $x \in U \cap W$, then $x = u + w$ uniquely, with $u \in U$, $w \in W$. But also $x - u = w \in U$, so $x \in U \Rightarrow w \in U$, hence $w \in U \cap W$. So $x = 0$.

(\( \Leftarrow \)) Suppose $x = x_1 + y_1 = x_2 + y_2$ with $x_i \in U$, $y_i \in W$. Then $x_1 - x_2 = y_2 - y_1 \in U \cap W$. Since intersection is \{0\}, we get $x_1 = x_2$ and $y_1 = y_2$.
\end{proof}

\section{Finite-Dimensional Vector Spaces}

A linear combination of $v_1, \dots, v_m \in V$ is any vector
\[
  a_1v_1 + \dots + a_mv_m
\]
The span is
\[
  \text{span}(v_1, \dots, v_m) = \left\{ a_1v_1 + \dots + a_mv_m : a_i \in \mathbb{F} \right\}
\]

\textbf{Note:} The span of a list of vectors is the smallest subspace containing them. If $\text{span}(v_1, \dots, v_m) = V$, we say the list \textit{spans} $V$.

A list $v_1, \dots, v_m$ is linearly independent if the only solution to
\[
  a_1v_1 + \dots + a_mv_m = 0
\]
is $a_1 = \dots = a_m = 0$.

\begin{lemma}
If $v_1, \dots, v_m$ is a linearly dependent list, there exists $k$ such that $v_k \in \text{span}(v_1, \dots, v_{k-1})$. Removing $v_k$ does not change the span.
\end{lemma}

\begin{theorem}
In a finite-dimensional vector space, every linearly independent list has length $\leq$ any spanning list.
\end{theorem}
\begin{proof}
Let $A = (v_1, \dots, v_m)$ be linearly independent, $B = (w_1, \dots, w_n)$ spans $V$. Add $v_1$ to $B$: since $B$ spans $V$, $v_1$ is a combination of $w_i$, making the new list dependent. Remove a dependent $w_i$, repeat for each $v_i$. Since $B$ has $n$ vectors, we cannot insert more than $n$ independent $v_i$, so $m \leq n$.
\end{proof}

\begin{theorem}
Every subspace of a finite-dimensional vector space is finite-dimensional.
\end{theorem}
\begin{proof}
Let $V$ finite-dimensional, $U \subseteq V$ a subspace. If $U = \{0\}$, done. Otherwise pick $u_1 \in U$ nonzero. Inductively build linearly independent list $\{u_1, \dots, u_k\} \subset U$.

If $U \subset \text{span}(u_1, \dots, u_k)$, done. Otherwise pick $u_{k+1} \notin \text{span}(\dots)$. Since $V$ is finite-dimensional, this process stops after $\dim V$ steps. Hence $U$ is spanned by finitely many vectors.
\end{proof}

\textbf{Definition (Basis):} A basis of $V$ is a linearly independent list that spans $V$.

Every $v \in V$ has a unique representation in a basis:
\[
  v = a_1v_1 + \dots + a_mv_m
\]

Every spanning list can be reduced to a basis. Every finite-dimensional space has a basis. Every linearly independent list can be extended to a basis.

\end{document}
